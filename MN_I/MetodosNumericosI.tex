%%%%%%%%%%%%%%%%%%%%%%%%%%%%%%%%%%%%%%%%%%%%%%%%%%%%%%%%%%%%%%%%%%%%%%%%%
%
% Plantilla para libro de texto de matemáticas.
%
% Esta plantilla ha sido desarrollada desde cero, pero utiliza algunas partes
% del código de la plantilla original utilizada en apuntesDGIIM
% (https://github.com/libreim/apuntesDGIIM), basada a su vez en las plantillas
% 'Short Sectioned Assignment' de Frits Wenneker (http://www.howtotex.com),
% 'Plantilla de Trabajo' de Mario Román y 'Plantilla básica de Latex en Español'
% de Andrés Herrera Poyatos (https://github.com/andreshp). También recoge
% ideas de la plantilla 'Multi-Purpose Large Font Title Page' de
% Frits Wenneker y Vel (vel@latextemplates.com).
%
% Licencia:
% CC BY-NC-SA 4.0 (https://creativecommons.org/licenses/by-nc-sa/4.0/)
%
%%%%%%%%%%%%%%%%%%%%%%%%%%%%%%%%%%%%%%%%%%%%%%%%%%%%%%%%%%%%%%%%%%%%%%%%%

% ---------------------------------------------------------------------------
% CONFIGURACIÓN BÁSICA DEL DOCUMENTO
% ---------------------------------------------------------------------------

%\documentclass[11pt, a4paper, twoside]{article} % Usar para imprimir
\documentclass[10pt, a4paper]{article}

\linespread{1.3}            % Espaciado entre líneas.
\setlength\parindent{0pt}   % No indentar el texto por defecto.
\setlength\parskip{7pt}

% ---------------------------------------------------------------------------
% PAQUETES BÁSICOS
% ---------------------------------------------------------------------------

% IDIOMA
\usepackage[utf8]{inputenc}
\usepackage[spanish, es-tabla, es-lcroman, es-noquoting]{babel}

% MATEMÁTICAS
\usepackage{amsmath}    % Paquete básico de matemáticas
\usepackage{amsthm}     % Teoremas
\usepackage{mathrsfs}   % Fuente para ciertas letras utilizadas en matemáticas

% FUENTES
\usepackage{newpxtext, newpxmath}   % Fuente similar a Palatino
\usepackage{FiraSans}                 % Fuente sans serif
\usepackage[T1]{fontenc}
\usepackage[italic]{mathastext}     % Utiliza la fuente del documento
                                    % en los entornos matemáticos

% MÁRGENES
\usepackage[margin=2.5cm, top=3cm]{geometry}

% LISTAS
\usepackage{enumitem}       % Mejores listas
\setlist{leftmargin=.5in}   % Especifica la indentación para las listas.


%Va en config:
% Listas ordenadas con números romanos (i), (ii), etc.
\newenvironment{nlist}
{\begin{enumerate}
    \renewcommand\labelenumi{(\emph{\roman{enumi})}}}
{\end{enumerate}}
  
%  OTROS
\usepackage{hyperref}   % Enlaces
\usepackage{graphicx}   % Permite incluir gráficos en el documento


% ---------------------------------------------------------------------------
% COLORES
% ---------------------------------------------------------------------------

\usepackage{xcolor}     % Permite definir y utilizar colores

\definecolor{50}{HTML}{FDDBDB}
\definecolor{100}{HTML}{FBC9C9}
\definecolor{200}{HTML}{FDB0B0}
\definecolor{300}{HTML}{F9A1A1}
\definecolor{400}{HTML}{F98F8F}
\definecolor{500}{HTML}{F67A7A}
\definecolor{600}{HTML}{F76565}
\definecolor{700}{HTML}{FB4D4D}
\definecolor{800}{HTML}{C50000}
\definecolor{900}{HTML}{920101}

% ---------------------------------------------------------------------------
% DISEÑO DE PÁGINA
% ---------------------------------------------------------------------------

\usepackage{pagecolor}
\usepackage{afterpage}

% ---------------------------------------------------------------------------
% CABECERA Y PIE DE PÁGINA
% ---------------------------------------------------------------------------

\usepackage{fancyhdr}   % Paquete para cabeceras y pies de página

% Indica que las páginas usarán la configuración de fancyhdr
\pagestyle{fancy}
\fancyhf{}

% Representa la sección de la cabecera
\renewcommand{\sectionmark}[1]{%
\markboth{#1}{}}

% Representa la subsección de la cabecera
\renewcommand{\subsectionmark}[1]{%
\markright{#1}{}}

% Parte derecha de la cabecera
\fancyhead[LE,RO]{\sffamily\textsl{\rightmark} \hspace{1em}  \textcolor{500}{\rule[-0.4ex]{0.2ex}{1.2em}} \hspace{1em} \thepage}

% Parte izquierda de la cabecera
\fancyhead[RE,LO]{\sffamily{\leftmark}}

% Elimina la línea de la cabecera
\renewcommand{\headrulewidth}{0pt}

% Controla la altura de la cabecera para que no haya errores
\setlength{\headheight}{14pt}

% ---------------------------------------------------------------------------
% TÍTULOS DE PARTES Y SECCIONES
% ---------------------------------------------------------------------------

\usepackage{titlesec}

% Estilo de los títulos de las partes
\titleformat{\part}[hang]{\Huge\bfseries\sffamily}{\thepart\hspace{20pt}\textcolor{500}{|}\hspace{20pt}}{0pt}{\Huge\bfseries}
\titlespacing*{\part}{0cm}{-2em}{2em}[0pt]

% Reiniciamos el contador de secciones entre partes (opcional)
\makeatletter
\@addtoreset{section}{part}
\makeatother

% Estilo de los títulos de las secciones, subsecciones y subsubsecciones
\titleformat{\section}
  {\Large\bfseries\sffamily}{\thesection}{1em}{}

\titleformat{\subsection}
  {\Large\sffamily}{\thesubsection}{1em}{}[\vspace{.5em}]

\titleformat{\subsubsection}
  {\sffamily}{\thesubsubsection}{1em}{}

% ---------------------------------------------------------------------------
% ENTORNOS PERSONALIZADOS
% ---------------------------------------------------------------------------

\usepackage{mdframed}

%% DEFINICIONES DE LOS ESTILOS

% Nuevo estilo para definiciones
\newtheoremstyle{definition-style}  % Nombre del estilo
{}                                  % Espacio por encima
{}                                  % Espacio por debajo
{}                                  % Fuente del cuerpo
{}                                  % Identación
{\bf\sffamily}                      % Fuente para la cabecera
{.}                                 % Puntuación tras la cabecera
{.5em}                              % Espacio tras la cabecera
{\thmname{#1}\thmnumber{ #2}\thmnote{ (#3)}}  % Especificación de la cabecera

% Nuevo estilo para notas
\newtheoremstyle{remark-style}
{10pt}
{10pt}
{}
{}
{\itshape \sffamily}
{.}
{.5em}
{}

% Nuevo estilo para teoremas y proposiciones
\newtheoremstyle{theorem-style}
{}
{}
{}
{}
{\bfseries \sffamily}
{.}
{.5em}
{\thmname{#1}\thmnumber{ #2}\thmnote{ (#3)}}

% Nuevo estilo para ejemplos
\newtheoremstyle{example-style}
{10pt}
{10pt}
{}
{}
{\bf \sffamily}
{}
{.5em}
{\thmname{#1}\thmnumber{ #2.}\thmnote{ #3.}}

% Nuevo estilo para la demostración

\makeatletter
\renewenvironment{proof}[1][\proofname] {\par\pushQED{\qed}\normalfont\topsep6\p@\@plus6\p@\relax\trivlist\item[\hskip\labelsep\itshape\sffamily#1\@addpunct{.}]\ignorespaces}{\popQED\endtrivlist\@endpefalse}
\makeatother

%% ASIGNACIÓN DE LOS ESTILOS

% Teoremas, proposiciones y corolarios
\theoremstyle{theorem-style}
\newtheorem{nth}{Teorema}[section]
\newtheorem{nprop}{Proposición}[section]
\newtheorem{ncor}{Corolario}[section]
\newtheorem{lema}{Lema}[section]

% Definiciones
\theoremstyle{definition-style}
\newtheorem{ndef}{Definición}[section]

% Notas
\theoremstyle{remark-style}
\newtheorem*{nota}{Nota}

% Ejemplos
\theoremstyle{example-style}
\newtheorem{ejemplo}{Ejemplo}[section]

% Ejercicios y solución
\theoremstyle{definition-style}
\newtheorem{ejer}{Ejercicio}[section]

\theoremstyle{remark-style}
\newtheorem*{sol}{Solución}

%% MARCOS DE LOS ESTILOS

% Configuración general de mdframe, los estilos de los teoremas, etc
\mdfsetup{
  skipabove=1em,
  skipbelow=1em,
  innertopmargin=1em,
  innerbottommargin=1em,
  splittopskip=2\topsep,
}

% Definimos los marcos de los estilos

\mdfdefinestyle{nth-frame}{
	linewidth=2pt, %
	linecolor= 500, %
	topline=false, %
	bottomline=false, %
	rightline=false,%
	leftmargin=0em, %
	innerleftmargin=1em, %
  innerrightmargin=1em,
	rightmargin=0em, %
}%

\mdfdefinestyle{nprop-frame}{
	linewidth=2pt, %
	linecolor= 300, %
	topline=false, %
	bottomline=false, %
	rightline=false,%
	leftmargin=0pt, %
	innerleftmargin=1em, %
	innerrightmargin=1em,
	rightmargin=0pt, %
}%

\mdfdefinestyle{ndef-frame}{
	linewidth=2pt, %
	linecolor= 500, %
	backgroundcolor= 50,
	topline=false, %
	bottomline=false, %
	rightline=false,%
	leftmargin=0pt, %
	innerleftmargin=1em, %
	innerrightmargin=1em,
	rightmargin=0pt, %
}%

\mdfdefinestyle{ejer-frame}{
	linewidth=2pt, %
	linecolor= 300, %
	backgroundcolor= 50,
	topline=false, %
	bottomline=false, %
	rightline=false,%
	leftmargin=0pt, %
	innerleftmargin=1em, %
	innerrightmargin=1em,
	rightmargin=0pt, %
}%

\mdfdefinestyle{ejemplo-frame}{
	linewidth=0pt, %
	linecolor= 300, %
	leftline=false, %
	rightline=false, %
	leftmargin=0pt, %
	innerleftmargin=1.3em, %
	innerrightmargin=1em,
	rightmargin=0pt, %
	innertopmargin=0em,%
	innerbottommargin=0em, %
	splittopskip=\topskip, %
}%

% Asignamos los marcos a los estilos
\surroundwithmdframed[style=nth-frame]{nth}
\surroundwithmdframed[style=nprop-frame]{nprop}
\surroundwithmdframed[style=nprop-frame]{ncor}
\surroundwithmdframed[style=ndef-frame]{ndef}
\surroundwithmdframed[style=ejer-frame]{ejer}
\surroundwithmdframed[style=ejemplo-frame]{ejemplo}
\surroundwithmdframed[style=ejemplo-frame]{sol}

% ---------------------------------------------------------------------------
% CONFIGURACIÓN DE LA PORTADA
% ---------------------------------------------------------------------------

\newcommand{\asignatura}{%
  Métodos\\
  numéricos I\\
}

\newcommand{\autor}{LibreIM}

\newcommand{\grado}{Doble grado de ingeniería informática y matemáticas}

\newcommand{\universidad}{Universidad de Granada}

\newcommand{\enlaceweb}{libreim.github.io}

% ---------------------------------------------------------------------------
% CONFIGURACIÓN PERSONALIZADA
% ---------------------------------------------------------------------------

%%%%%%%%%%%%%%%%%%%%%%%%%%%%%%%%%%%%%%%%%%%%%%%%%%%%%%%%%%%%%%%%%%%%%%%%%%%%%
% ---------------------------------------------------------------------------
% COMIENZO DEL DOCUMENTO
% ---------------------------------------------------------------------------
%%%%%%%%%%%%%%%%%%%%%%%%%%%%%%%%%%%%%%%%%%%%%%%%%%%%%%%%%%%%%%%%%%%%%%%%%%%%%

\begin{document}

% ---------------------------------------------------------------------------
% PORTADA EXTERIOR
% ---------------------------------------------------------------------------

\newpagecolor{500}\afterpage{\restorepagecolor} % Color de la página
\begin{titlepage}

  % Título del documento
	\parbox[t]{\textwidth}{
			\raggedright % Texto alineado a la izquierda
			\fontsize{50pt}{50pt}\selectfont\sffamily\color{white}{
			  \textbf{\asignatura}
      }
	}

	\vfill

	%% Autor e información del documento
	\parbox[t]{\textwidth}{
		\raggedright % Texto alineado a la izquierda
		\sffamily\large\color{white}
		{\Large \autor }\\[4pt]
		\grado\\
		\universidad\\[4pt]
		\texttt{\enlaceweb}
	}

\end{titlepage}

% ---------------------------------------------------------------------------
% PÁGINA DE LICENCIA
% ---------------------------------------------------------------------------

\thispagestyle{empty}
\null
\vfill

%% Información sobre la licencia
\parbox[t]{\textwidth}{
  \includegraphics{by-nc-sa.pdf}\\[4pt]
  \raggedright % Texto alineado a la izquierda
  \sffamily\large
  {\Large Este libro se distribuye bajo una licencia CC BY-NC-SA 4.0.}\\[4pt]
  Eres libre de distribuir y adaptar el material siempre que reconozcas a los\\
  autores originales del documento, no lo utilices para fines comerciales\\
  y lo distribuyas bajo la misma licencia.\\[4pt]
  \texttt{creativecommons.org/licenses/by-nc-sa/4.0/}
}

% ---------------------------------------------------------------------------
% PORTADA INTERIOR
% ---------------------------------------------------------------------------

\begin{titlepage}

  % Título del documento
	\parbox[t]{\textwidth}{
			\raggedright % Texto alineado a la izquierda
			\fontsize{50pt}{50pt}\selectfont\sffamily\color{500}{
			  \textbf{\asignatura}
      }
	}

	\vfill

	%% Autor e información del documento
	\parbox[t]{\textwidth}{
		\raggedright % Texto alineado a la izquierda
		\sffamily\large
		{\Large \autor}\\[4pt]
		\grado\\
		\universidad\\[4pt]
		\texttt{\enlaceweb}
	}

\end{titlepage}

% ---------------------------------------------------------------------------
% ÍNDICE
% ---------------------------------------------------------------------------

\thispagestyle{empty}
\tableofcontents
\newpage

% ---------------------------------------------------------------------------
% CONTENIDO
% ---------------------------------------------------------------------------

\part{Tema 1. Introducción a los problemas del Análisis Numérico}


\section{Introducción a los métodos numéricos: algoritmo}
Comenzaremos con un ejemplo de recurrencia en el que observaremos que al redondear el primer valor, se acumula el error y los siguientes valores se desbordan.\\
Sea n $\geq$ 1 y la recurrencia $x_{n}$:=$\int_{0}^{1} x^{n}e^{x}dx$.\\
Si resolvemos la integral para n = 0, tenemos que $x_{0}$ = e - 1\\
Y para n $\in$ $\mathbb{N}$, tenemos que $x_{n}$ = e - n$x_{n-1}$\\
Por lo que esta sucesión, $\lbrace x_{n} \rbrace_{n\geq1}$ $\subset$ $\mathbb{R_{+}}$, es decreciente y tiende a 0, es decir, $ \lim_{n \to \infty} x_{n} = 0 $ \\
Veamos que si redondeamos $x_{0}$ se acumula el error.\\
Si n = 12 tenemos que $x_{12}$ = 0.1951\\
Redondeando $x_{0}$ = 1.7183 e iterando según este valor hasta n = 12, obtenemos que $x_{12}$ = 8704.39\\
Luego, este valor es, con diferencia, mayor que el que habíamos calculado sin redondeo y $x_{n}$ no tiende a 0.\\
Concluimos que el redondeo, a veces, conlleva errores muy grandes.

\subsection{Espacios normados}
Si usamos las normas en los problemas numéricos, sabremos si los problemas están bien planteados, los errores cometidos, la convergencia...\\

\begin{ndef}[Norma]
Sea E un espacio vectorial real, diremos que una aplicacion $\Vert$·$\Vert$: E $\rightarrow$ $\mathbb{R}$ es una $\textbf{norma}$ en E si verifica las siguientes propiedades:
	\begin{nlist}
	\item Sea x $\in$ E $\Rightarrow$ $\Vert$·$\Vert$ $\geq$ 0.\\
	Además, $\Vert$x$\Vert$ = 0 $\Leftrightarrow$ x = 0
	\item Sean x,y $\in$ E $\Rightarrow$ $\Vert$x + y$\Vert$ $\leq$ $\Vert$x$\Vert$ + $\Vert$y$\Vert$ $\quad$ (desigualdad triangular)
	\item Sean x $\in$ E, $\lambda$ $\in$ $\mathbb{R}$ $\Rightarrow$ $\Vert$ $	\lambda$x $\Vert$ = $\vert$ $\lambda$ $\vert$ $\Vert$x$\Vert$
	\end{nlist}
\end{ndef}

\begin{ndef}[Espacio normado]
Sea E un espacio vectorial real. Si este espacio admite una norma, entonces E se llama $\textbf{espacio normado}$.
\end{ndef}

Aunque trabajaremos en $\mathbb{R}$, en $\mathbb{C}$ es lo mismo.\\

Veamos la interpretación geométrica de la desigualdad triangular, usando la norma euclídea $\mathbb{R}^2$ o $\mathbb{R}^3$.\\
\includegraphics[scale=0.2]{media/desigualdadtriangular.png}

Las siguientes normas son las que vamos a utilizar.

\begin{ndef}[Norma p]
Sean E = $\mathbb{R}^N$, p $\geq$ 1 y x $\in$ $\mathbb{R}^N$, entonces:
\[ \Vert x \Vert _{p} := \left( \sum_{j=1}^{N} \vert x_{j} \vert ^p \right) ^{1/p} \]
Si p = 2, entonces la norma es $\textbf{euclídea}$.
\end{ndef}

\begin{ndef}[Norma del máximo]
Sea E = $\mathbb{R}^N$ y x $\in$ $\mathbb{R}^N$, entonces:
\[ \Vert x \Vert _{\infty} := max \lbrace \vert x_{j} \vert : j = 1,...,N \rbrace \]
\end{ndef}

\begin{ndef}[Norma de Frobenius]
Sea E = $\mathbb{R}^{M \times N}$ y A $\in$ $\mathbb{R}^{M \times N}$, entonces:
\[ \Vert A \Vert _F := \sqrt{\sum_{i=1}^{M} \sum_{j=1}^{N} a_{ij}^2} \]
\end{ndef}

$\textit{Aclaración:}$\\
$\textit{Si estamos en un espacio vectorial real C([a,b]), esto significa que este espacio está compuesto por todas las funciones}\\
\textit{continuas en el intervalo cerrado [a,b].}$\\
$\textit{Si el espacio es $C^k$([a,b]), significa que está formado por las funciones de clase k, es decir, funciones derivables hasta}\\
\textit{orden k y esas derivadas son continuas.}$

\begin{ndef}[Norma del máximo]
Sea E = C([a,b]) y f $\in$ C([a,b]), entonces:
\[ \Vert f \Vert _\infty := max \; \left\lbrace \vert f(x) \vert : a \leq x \leq b \right\rbrace \]
\end{ndef}

\begin{ndef}
Sean E = $C^k$([a,b]), k $\in$ $\mathbb{N}$ y f $\in$ $C^k$([a,b]) entonces:
\[ \Vert f \Vert _k := max \; \left\lbrace \Vert f^{(j)} \Vert : j = 0,...,k \right\rbrace \]
\end{ndef}

Ahora que ya tenemos definidas las normas, podemos calcular el error cometido al aproximar los vectores.

\begin{ndef}[Error absoluto]
Sea E un espacio normado, x $\in$ E y x* $\in$ una aproximación de x, entonces la siguiente operación calcula el error absoluto:
\[ \Vert x^* - x \Vert \]
\end{ndef}

\begin{ndef}[Error relativo]
Sea E un espacio normado, x $\in$ E y x* $\in$ una aproximación de x, entonces la siguiente operación calcula el error relativo:
\[  \frac {\Vert x^* - x \Vert}{ \Vert x \Vert} \]
\end{ndef}

Veamos una aplicación de estos errores.

\begin{ejer}
Calcula los errores absolutos y relativos de:
	\begin{nlist}
	\item E = $\mathbb{R}$, x = 1/4, x* = 0.23
	\item E = $\mathbb{R}^3$, x = (1/5,2,1), x* = (0.19,2.2,0.9)
	\item E = C([0,$\pi$/2]), f(t) = sen(t), f*(t) = t
	\end{nlist}
\end{ejer}

\begin{sol} $\newline$
	\begin{nlist}
	\item error absoluto: $\vert x^* - x \vert = \vert 0.23 - 1/4 \vert = 0.02 $\\
	error relativo: $\frac{\vert x^* - x \vert}{\vert x \vert} = \frac{\vert 0.23 - 1/4 \vert}{\vert 1/4 \vert} = 0.08$
	\item error absoluto: $\Vert x^* - x \Vert _\infty = \Vert (0.19,2.2,0.9) - (1/5,2,1) \Vert _\infty = \Vert (-0.01,0.2,-0.1) \Vert _\infty =$\\ $= max \lbrace 0.01,0.2,0.1\rbrace = 0.2 $\\
	error relativo: $\frac{\Vert x^* - x \Vert _\infty}{\Vert x \Vert _\infty} = \frac{\Vert (-0.01,0.2,-0.1) \Vert _\infty}{\Vert (1/5,2,1) \Vert _\infty} = \frac{0.2}{2} = 0.1 $\\
	\item error absoluto: $\Vert f^* - f \Vert _\infty = \Vert t - sen(t) \Vert _\infty = \frac{\pi}{2} - 1$\\
	error relativo: $\frac{\Vert f^* - f \Vert _\infty}{\Vert f \Vert _\infty} = \frac{\pi}{2} - 1 $
	\end{nlist}
\end{sol}

\begin{ndef}[Distancia]
Se define la $\textbf{distancia}$ entre dos vectores x,y $\in$ E como
\[ dist(x,y) := \Vert x - y \Vert \]
\end{ndef}

\begin{ndef}
Se dice que $\lbrace x_n \rbrace _{n \geq 1}$ en E $\textbf{converge}$ a $x_0$ $\in$ E sii
\[ \forall \varepsilon > 0 \Rightarrow \left[ \exists n_0 \in \mathbb{N} : n \geq n_0 \Rightarrow \Vert x_n - x_0 \Vert < \varepsilon \right] \]
es decir,
\[ \lim_{n \rightarrow \infty} x_n = x_0  \Leftrightarrow \lim_{n \rightarrow \infty} \Vert x_n - x_0 \Vert = 0 \]
\end{ndef}

\begin{ndef}
Sean X, Y subconjuntos no vacíos de sendos espacios normados y sea f: X $\rightarrow$ Y, diremos que f es $\textbf{continua}$ en $x_{0}$ $\in$ X si
\[ \forall \varepsilon > 0 \Rightarrow \left[ \exists \delta > 0 : x \in X \wedge \Vert x - x_0 \Vert < \delta \Rightarrow \Vert f(x) - f(x_0) \Vert < \varepsilon \right] \]
\end{ndef}

\begin{nprop}
Sea x $\in \mathbb{R}^N \Rightarrow \Vert x \Vert _\infty \leq \Vert x \Vert _1 \leq N\Vert x \Vert _\infty $ 
\end{nprop}

\begin{ndef}
Sean $\Vert$ · $\Vert$ y $\Vert$ · $\Vert _*$ dos normas, se dice que son $\textbf{equivalentes}$ si $\exists c_1, c_2 > 0$ tales que
\[ \forall x \in E \Rightarrow c_1\Vert x \Vert \leq \Vert x \Vert _* \leq c_2\Vert x \Vert \] 
\end{ndef}

\begin{nprop}
Sean $\Vert$ · $\Vert$ y $\Vert$ · $\Vert _*$ dos normas, entonces la convergencia de sucesiones y la continuidad son equivalentes para ambas normas. 
\end{nprop}

\begin{nth}
Todas las normas en un espacio normado finito dimensional son equivalentes.
\end{nth}

Observemos que para calcular el límite de la norma del máximo, tenemos que calcular el límite de cada coordenada.

\begin{nprop}
Sea $\mathbb{R}^N$ un espacio normado finito dimensional y consideremos la norma $\Vert$·$\Vert _\infty$ en este espacio, entonces:
\[ \lim_{n \rightarrow \infty}x_n = x_0 \Leftrightarrow \lim_{n \geq 1}(x_n)_j = (x_0)_j \qquad \forall j\in \lbrace 1,...,N \rbrace \]
\end{nprop}

\begin{proof}
\[ \lim_{n \rightarrow \infty} x_n = x_0 \quad \Leftrightarrow \quad \lim_{n \rightarrow \infty} \Vert x_n - x_0 \Vert _\infty = 0 \quad \Leftrightarrow \quad 0 < max \; \lbrace \vert (x_n - x_0)_j \vert : j = 1,...,N \rbrace = 0 \quad \rightarrow \quad \lim_{n \geq 1}(x_n)_j = (x_0)_j \]
\end{proof}

Un ejemplo de aplicación de esta proposición es el siguiente.

\begin{ejemplo}
$\lim_{n \geq 1} \left( \left( 1 + \frac{1}{n} \right) ^n , \frac{(-1)^n}{n^2} \right) = (e,0)$
\end{ejemplo}

La anterior proposición también se puede aplicar para cualquier norma de $\mathbb{R}^N$ y de $\mathbb{R}^{M \times N}$.

\begin{ejer}
Comprueba que la norma del máximo en C([0,1]) no es equivalente a la norma $\Vert$·$\Vert _1$ definida para cada f $\in$ C([0,1]) como
\[ \Vert f \Vert _1 := \int_0^1 \vert f(x) \vert dx \]
(Indicación: para cada n $\geq$ 2, considera la función $f_n$ cuya gráfica es la poligonal que une los puntos (0,0), (1/n,1), (2/n,0), (1,0)).
\end{ejer}

\begin{sol}
Tenemos que\\
\[ E = C([0,1]) \]
\[ \Vert f \Vert _\infty = max \left\lbrace \vert f(x) \vert : 0 \leq x \leq 1 \right\rbrace \]
\[ \Vert f \Vert _1 := \int_0^1 \vert f(x) \vert dx \]
Luego\\
$\Vert f_n \Vert _\infty = 1$ y $\Vert f_n \Vert _1 = 1/n$ (que coincide con el área).\\
Si fueran equivalentes, entonces
\[ \exists \alpha , \beta > 0 f \in E \Rightarrow \alpha \Vert f_n \Vert _1 \leq \Vert f_n \Vert _\infty \leq \beta \Vert f_n \Vert _1 \]
Lo cual es una contradicción, porque n $\geq$ 1 $\Rightarrow \Vert f_n \Vert _\infty \leq \beta \Vert f_n \Vert _1 \Leftrightarrow 1 \leq \frac{\beta}{n} \Leftrightarrow n \leq \beta$ $\;$ y n no está acotada.\\
Por lo que no son equivalentes.
\end{sol}

\begin{nprop}
Sean M, N $\in$ $\mathbb{N}$ y consideremos sendas normas en $\mathbb{R}^N$ y $\mathbb{R}^M$, que sin lugar a ambigüedad notaremos indeferentemente como $\Vert$·$\Vert$. Entonces la aplicación que notaremos igualmente como $\Vert$·$\Vert$ define una norma en $\mathbb{R}^{M \times N}$:
\[ \Vert A \Vert := sup \; \left\lbrace \Vert Ax \Vert : x \in \mathbb{R}^N \wedge \Vert x \Vert = 1 \right\rbrace \qquad \forall A \in \mathbb{R}^{M \times N} \]
\end{nprop}

\begin{ndef}
Se define la norma \textbf{inducida} en $\mathbb{R}^{M \times N}$ como:
\[ \Vert A \Vert := sup \; \left\lbrace \Vert Ax \Vert : x \in \mathbb{R}^N \wedge \Vert x \Vert = 1 \right\rbrace \qquad \forall A \in \mathbb{R}^{M \times N} \]
\end{ndef}

\begin{nprop}
Con la notación de la proposición anterior, si $A \in \mathbb{R}^{M \times N}$ entonces
\[ \Vert A \Vert := sup \; \left\lbrace \frac{\Vert Ax \Vert}{\Vert x \Vert } : x \in \mathbb{R}^N \wedge x \neq 0 \right\rbrace \]
En particular,
\[ \Vert Ax \Vert \leq \Vert A \Vert \Vert x \Vert \]
\end{nprop}

\begin{nprop}
Consideremos la norma $\Vert$·$\Vert _1$ en $\mathbb{R}^N$ y en $\mathbb{R}^M$, entonces la norma $\Vert$·$\Vert _1$ inducida en $\mathbb{R}^{M \times N}$ es
\[ \Vert A \Vert _1 = max \; \left\lbrace \sum_{i=1}^M \vert a_{ij} \vert : j = 1,...,N \right\rbrace \qquad \forall A \in \mathbb{R}^{M \times N} \]
\end{nprop}

Es decir, es el máximo de las sumas de los valores absolutos de cada $\textbf{columna}$.

\begin{proof} Vamos a demostrar que es $\geq$ y $\leq$, luego se dará la igualdad.\\
Probaremos primero que $ \Vert A \Vert _\infty \geq max \; \left\lbrace \sum_{j=1}^N \vert a_{ij} \vert : i = 1,...,M \right\rbrace $\\
Sea sign(a) := $\left\{ \begin{array}{lcc}
-1 & si & a < 0 \\
1 & si & a \geq 0
\end{array}
\right.$ $\qquad$ $\forall a \in \mathbb{R}$\\
Tenemos que
\[ \left\Vert \left[ sign(a_{11}),...,sign(a_{1N}) \right] ^T \right\Vert _\infty = 1 \qquad \Rightarrow \qquad \Vert A \Vert _\infty \geq \left\Vert A \left[ sign(a_{11}),...,sign(a_{1N}) \right] ^T \right\Vert _\infty \geq \sum_{j=1}^N \vert a_{1j} \vert \]
Hacemos lo mismo con $\left[ sign(a_{i1}),...,sign(a_{iN}) \right] ^T \; \forall i = 2,...,M$ y obtenemos que
\[ \Vert A \Vert _\infty \geq max \; \left\lbrace \sum_{j=1}^N \vert a_{ij} \vert : i = 1,...,M \right\rbrace \]
Ahora probaremos que $ \Vert A \Vert _\infty \leq max \; \left\lbrace \sum_{j=1}^N \vert a_{ij} \vert : i = 1,...,M \right\rbrace $\\
Sea x $\in \mathbb{R}^N$ tal que $\Vert x \Vert _\infty = 1$, entonces:
\[ \Vert Ax \Vert _\infty = \left\Vert 
\begin{bmatrix} 
a_{11} &  \cdots & a_{1N} \\
\vdots & & \vdots 
\\ a_{M1} & \cdots & a_{MN} \\ \end{bmatrix} 
\begin{bmatrix}
x_1 \\
\vdots \\
x_N \\
\end{bmatrix}
\right\Vert _\infty = \left\Vert 
\begin{bmatrix}
\sum_{j=1}^N a_{1j}x_j & ,\ldots , & \sum_{j=1}^N a_{Mj}x_j 
\end{bmatrix} ^T
\right\Vert _\infty = \] \[= max \; \left\lbrace \vert \sum_{j=1}^N a_{ij}x_j \vert : i = 1,...,M \right\rbrace \leq max \; \left\lbrace \sum_{j=1}^N \vert a_{ij} \vert \vert x_j \vert : i = 1,...,M \right\rbrace \leq \] \[ \leq max \; \left\lbrace \sum_{j=1}^N \vert a_{ij} \vert : i = 1,...,M \right\rbrace \] 
\end{proof}

\begin{nprop}
Consideremos la norma $\Vert$·$\Vert _\infty$ en $\mathbb{R}^N$ y en $\mathbb{R}^M$, entonces la norma $\Vert$·$\Vert _\infty$ inducida en $\mathbb{R}^{M \times N}$ es
\[ \Vert A \Vert _\infty = max \; \left\lbrace \sum_{j=1}^N \vert a_{ij} \vert : i = 1,...,N \right\rbrace \qquad \forall A \in \mathbb{R}^{M \times N} \]
\end{nprop}

Es decir, es el máximo de las sumas de los valores absolutos de cada $\textbf{fila}$.\\

Por lo que si ya hemos calculado alguna de estas dos últimas normas, podemos saber la otra sin tener que volver a calcular el máximo, es decir, la relación entre ambas viene en la siguiente proposición.

\begin{nprop}
$\Vert A \Vert _1 = \Vert A^T \Vert _\infty \qquad \forall A \in \mathbb{R}^{M \times N}$
\end{nprop}

Hay que tener en cuenta que $\Vert$ · $\Vert _2$ no induce en $\mathbb{R}^{M \times N}$ Frobenius.\\

Además, para las matrices $\textbf{cuadradas}$ tenemos la siguiente definición.

\begin{ndef}[Radio espectral]
Sea $A \in \mathbb{R}^{N \times N}$, denotaremos como $\textbf{radio espectral de A}$ a:
\[ \rho (A) := max \; \left\lbrace \vert \lambda \vert : \lambda \in \mathbb{C} \wedge det(A - \lambda I) = 0 \right\rbrace \]
\end{ndef}

La siguiente proposición muestra una manera más fácil de calcular la norma euclídea de un vector de $\mathbb{R}^N$, que es calculando la suma de las coordenadas al cuadrado.

\begin{nprop}
$\Vert x \Vert _2 = \sqrt{x^Tx} \qquad \forall x \in \mathbb{R}^N$
\end{nprop}

\begin{ndef}
Sea A $\in \mathbb{R}^{N \times N}$, diremos que A es $\textbf{semidefinida positiva} \Leftrightarrow x^TAx \geq 0 \qquad \forall x \in \mathbb{R}^N $
\end{ndef}

Las matrices semidefinidas positivas tienen la siguiente propiedad.

\begin{nprop}
Sea A $\in \mathbb{R}^{N \times N}$ semidefinida positiva. Si $\lambda$ es un valor propio de A $\Rightarrow \lambda \geq 0$.
\end{nprop}

\begin{proof}
Como $\lambda$ es valor propio de A $\Rightarrow \exists x \in \mathbb{R}^N : x \neq 0 \wedge Ax = \lambda x$\\
Luego\\
\[ 0 \leq x^TAx = x^T \lambda x = \lambda x^Tx = \lambda \Vert x \Vert _2^2 \]
Como x $\neq$ 0 $\Rightarrow 0 \leq \lambda$
\end{proof}

\begin{nprop}
Sea P $\in \mathbb{R}^{N \times N}$ una matriz ortogonal, entonces
\[ \left\lbrace x \in \mathbb{R}^N : \Vert x \Vert _2 = 1 \right\rbrace = \left\lbrace P^Tx : x \in \mathbb{R}^N \wedge \Vert x \Vert _2 = 1 \right\rbrace \]
\end{nprop}

\begin{proof}
Vamos a demostrar la doble inclusión, lo que dará la igualdad.\\
$\textbf{¿ $\supseteq$ ?}$\\
Sea $x \in \mathbb{R}^N : \Vert x \Vert _2 = 1 \; \Rightarrow \;$ ¿ $\Vert P^Tx \Vert _2 = 1 $ ?
\[ \Vert P^Tx \Vert _2 = \sqrt{x^TPP^Tx} = \sqrt{x^Tx} = \Vert x \Vert _2 = 1 \]
$\textbf{¿ $\subseteq$ ?}$\\
\[ 1 = \Vert x \Vert _2 = \sqrt{x^Tx} = \sqrt{x^TIx} = \sqrt{x^TPP^Tx} = \Vert P^Tx \Vert _2 \]
\end{proof}

\begin{nprop}
Si $\lambda _1,..., \lambda _N \geq 0 \Rightarrow sup \; \left\lbrace \sqrt{\sum_{i=1}^N \lambda _iy_i^2} : y \in \mathbb{R}^N \wedge \Vert y \Vert _2 = 1 \right\rbrace = \sqrt{max \; \lambda _i : i = 1,...,N} $
\end{nprop}

\begin{proof}
\end{proof}

Una manera más sencilla de calcular la norma de una matriz es la siguiente.

\begin{nprop}
Sea A $\in \mathbb{R}^{M \times N} \Rightarrow \Vert A \Vert _2 = \sqrt{\rho (A^TA)} $
\end{nprop}

\begin{ndef}[Norma matricial]
Una norma en $\mathbb{R}^{N \times N}$ se dice $\textbf{matricial}$ cuando
\[ \Vert AB \Vert \leq \Vert A \Vert \Vert B \Vert \qquad \forall A,B \in \mathbb{R}^{N \times N} \]
\end{ndef}

Hay que tener en cuenta de que no toda norma en $\mathbb{R}^{N \times N}$ es matricial, por ejemplo:\\
Vamos a utilizar la siguiente norma\\
$\Vert A \Vert := max \; \lbrace \vert a{ij} \vert : i, j = 1,...,N \rbrace \qquad \forall A \in \mathbb{R}^{N \times N}$ \\
Sean A = B = $\begin{bmatrix}
1 & 1 \\
1 & 1 \\
\end{bmatrix}$\\
Luego tenemos que
$2 = \Vert AB \Vert > \Vert A \Vert \Vert B \Vert = 1$\\
Por lo que esta norma no es matricial.\\

\begin{nprop}
Toda norma en $\mathbb{R}^{N \times N}$ inducida por una norma en $\mathbb{R}^N$ es matricial.
\end{nprop}

\begin{proof}
Sea $\Vert$ · $\Vert$ una norma en $\mathbb{R}^{N \times N}$ inducida por una norma en $\mathbb{R}^N$, entonces
\[ \Vert A \Vert := sup \; \left\lbrace \Vert Ax \Vert : x \in \mathbb{R}^N \wedge \Vert x \Vert = 1 \right\rbrace \]
Sabemos que la norma es inducida, luego se cumple que
\[ \Vert Ax \Vert \leq \Vert A \Vert \Vert x \Vert \qquad \forall x \in \mathbb{R}^N \]
Tenemos que probar que $\Vert AB \Vert \leq \Vert A \Vert \Vert B \Vert$\\
\[ \Vert AB \Vert = sup \; \left\lbrace \Vert ABx \Vert : x \in \mathbb{R}^N \wedge \Vert x \Vert = 1 \right\rbrace \leq sup \; \left\lbrace \Vert A \Vert \Vert Bx \Vert : x \in \mathbb{R}^N \wedge \Vert x \Vert = 1 \right\rbrace \leq \] \[ \leq sup \; \left\lbrace \Vert A \Vert \Vert B \Vert \Vert x \Vert : x \in \mathbb{R}^N \wedge \Vert x \Vert = 1 \right\rbrace = \Vert A \Vert \Vert B \Vert \]
\end{proof}

\begin{nth}
$\lim_{n \rightarrow \infty} A^n = 0 \; \Leftrightarrow \; \rho (A) < 1 \qquad \forall A \in \mathbb{R}^{N \times N} $
\end{nth}

Este teorema nos deja dos importantes consecuencias.

\begin{ncor}
Sea A $\in \mathbb{R}^{N \times N}$ una matriz triangular, entonces
\[ \lim_{n \rightarrow \infty} A^n = 0 \quad \Leftrightarrow \quad max \; \lbrace \vert a_{ii} \vert < 1 : i = 1,...,N \rbrace \]
\end{ncor}

\begin{proof}
\end{proof}

\begin{ncor}
Sean N $\geq$ 1, A $\in \mathbb{R}^{N \times N}$ y $\Vert$ · $\Vert$ una norma matricial en $\mathbb{R}^{N \times N}$ tal que $\Vert$A$\Vert$ < 1, entonces $\rho$(A) < 1
\end{ncor}

Hay que tener en cuenta que no se cumple la implicación contraria, por ejemplo:\\
Sea A = $\begin{bmatrix}
0.5 & 500 \\
0 & 0.5 \\
\end{bmatrix}$ $\Rightarrow \rho (A) = 0.5 < 1$ pero $\Vert A \Vert _\infty = 500.5 \geq 1$


\subsection{Problemas bien planteados. Estabilidad}
Nos planteamos el siguiente problema:\\
Sean X e Y subconjuntos no vacíos de sendos espacios normados reales, f: X $\rightarrow$ Y una aplicación, $y_0 \in$ Y. Entonces tenemos que encontrar $x_0 \in X : f(x_0) = y_0$\\
Denotaremos a $x_0$ como la solución que resuelve el problema determinado por f y a $y_0$ los datos, es decir, son números. Si tenemos un conjunto finito de números, usaremos el vector de $\mathbb{R}^N$ o matriz, y si tenemos infinitos datos, usaremos una función.

\begin{ejemplo}
Sean $A \in \mathbb{R}^{M \times N}$ y $y \in \mathbb{R}^M$. Determinar una solución del sistema de ecuaciones lineales cuya matriz de coeficientes sea A y su vector de términos independientes sea y\\
$X = \mathbb{R}^N$, $Y = \mathbb{R}^M$, f(x) = Ax = y
\end{ejemplo}

\begin{ndef}
Un problema está $\textbf{bien planteado}$ cuando es $\textbf{unisolvente}$ y $\textbf{estable}$:
	\begin{nlist}
	\item $\exists ! x_0 \in X : f(x_0) = y_0$.
	\item $x_0$ depende continuamente de los datos $y_0$.
	\end{nlist}
\end{ndef}

En el siguiente ejemplo veremos un problema mal planteado.

\begin{ejemplo}
Sean X := $\mathbb{R}$, Y := $\mathbb{R}_+ \;$ y $\; f(x) := \vert x \vert , \; \forall x \in X$\\
Observamos que este problema no es unisolvente, ya que si $y_0$ = 1 (lo mismo vale $\forall y_0 > 0$) tenemos que f(-1) = 1 = f(1)
\end{ejemplo}

\begin{ndef}[Resolvente]
Denotaremos a la función g como la $\textbf{resolvente}$ de f si g es la inversa de f, para todo y $\in$ Y unisolvente.
\end{ndef}

\begin{ejemplo}
Sean X := $\mathbb{R}$, Y := $\mathbb{R}_+ \;$ y $\; f(x) := e^x , \; \forall x \in X$\\
Entonces este problema es unisolvente, luego tiene resolvente: g(y) = log y, para todo y $\in$ Y.
\end{ejemplo}

Podemos ver la estabilidad de un problema intuitivamente, es decir, a pequeñas perturbaciones de los datos $y_0$ corresponden pequeñas perturbaciones de la solución $x_0$.

\subsection{Algoritmos. Algoritmo PageRank de Google}


\section{Errores de redondeo. Iteradores}


\subsection{Sistema posicional y números máquina}


\subsection{Redondeo en sistemas de punto flotante y su aritmética}


\subsection{Iteradores}


\part{Tema2}


\end{document}